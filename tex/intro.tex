\documentclass[12pt,a4paper]{article}
\usepackage[utf8]{inputenc}
\usepackage[polish]{babel}
\usepackage[T1]{fontenc}
\usepackage{amsmath}
\usepackage{graphicx}
\usepackage{geometry}
\geometry{margin=2.5cm}

\title{Sprawozdanie wstępne:\\Cyfrowy magnetometr z wyświetlaczem OLED}
\author{}
\date{}

\begin{document}

\maketitle

\section{Cel projektu}

Celem projektu jest zbudowanie działającego magnetometru cyfrowego z wykorzystaniem mikrokontrolera STM32, sensora magnetycznego oraz wyświetlacza OLED. Urządzenie będzie mierzyć natężenie lokalnego pola magnetycznego w trzech osiach przestrzennych i prezentować wyniki w czasie rzeczywistym na ekranie.

\section{Wykorzystane komponenty sprzętowe}

\begin{itemize}
    \item \textbf{STM32 Nucleo-F446RE} -- płytka rozwojowa z mikrokontrolerem ARM Cortex-M4
    \item \textbf{GY-271 (QMC5883P)} -- trzyosiowy magnetometr cyfrowy
    \item \textbf{Wyświetlacz OLED 128×64} -- monochromatyczny ekran z kontrolerem SSD1306
    \item Przewody połączeniowe i płytka stykowa
\end{itemize}

\section{Podstawy fizyczne}

\subsection{Efekt Halla}

Projekt wykorzystuje czujnik magnetyczny oparty na \textbf{efekcie Halla}, który jest podstawowym zjawiskiem fizycznym umożliwiającym detekcję pola magnetycznego. Gdy przez przewodnik płynie prąd elektryczny $I$ w kierunku prostopadłym do zewnętrznego pola magnetycznego $\vec{B}$, na nośniki ładunku działa siła Lorentza:

\begin{equation}
\vec{F} = q(\vec{v} \times \vec{B})
\end{equation}

gdzie $q$ to ładunek nośnika, a $\vec{v}$ jego prędkość. Powoduje to nagromadzenie ładunków po jednej stronie przewodnika i powstanie napięcia Halla $V_H$ prostopadłego zarówno do kierunku prądu, jak i pola magnetycznego:

\begin{equation}
V_H = \frac{IB}{nqt}
\end{equation}

gdzie $n$ to koncentracja nośników ładunku, a $t$ grubość przewodnika. Pomiar tego napięcia pozwala na wyznaczenie natężenia pola magnetycznego.

\subsection{Pole magnetyczne Ziemi}

Magnetometr będzie głównie wykrywał \textbf{pole magnetyczne Ziemi}, które w Polsce ma natężenie około 50~$\mu$T (mikrotesli). Składa się ono z:

\begin{itemize}
    \item Składowej horyzontalnej $B_h \approx 20$~$\mu$T
    \item Składowej wertykalnej $B_v \approx 45$~$\mu$T
\end{itemize}

Nachylenie inklinacji magnetycznej w naszej lokalizacji wynosi około 67°.

\subsection{Komunikacja I²C}

Czujnik magnetyczny komunikuje się z mikrokontrolerem przez magistralę \textbf{I²C} (Inter-Integrated Circuit), wykorzystującą dwie linie:
\begin{itemize}
    \item SDA (Serial Data) -- linia danych
    \item SCL (Serial Clock) -- linia zegarowa
\end{itemize}

Transmisja danych odbywa się w sposób szeregowy, a stan logiczny 0 lub 1 jest reprezentowany przez różne poziomy napięcia zgodnie z prawami elektromagnetyzmu.

\subsection{Kalibracja i kompensacja zakłóceń}

Dokładny pomiar pola magnetycznego wymaga uwzględnienia:

\begin{itemize}
    \item \textbf{Zakłóceń twardych} (hard-iron) -- stałe przesunięcie pochodzące od elementów ferromagnetycznych w otoczeniu czujnika
    \item \textbf{Zakłóceń miękkich} (soft-iron) -- odkształcenie pola magnetycznego przez materiały ferromagnetyczne
\end{itemize}

Kalibracja polega na wyznaczeniu elipsoidy pomiarowej w przestrzeni 3D i przekształceniu jej na sferę poprzez operacje przesunięcia i skalowania:

\begin{equation}
\vec{B}_{kal} = S \cdot (\vec{B}_{raw} - \vec{O})
\end{equation}

gdzie $\vec{O}$ to wektor offsetów (hard-iron), a $S$ to macierz skal (soft-iron).

\subsection{Filtracja sygnału}

Ze względu na szumy pomiarowe, zastosowany zostanie \textbf{cyfrowy filtr dolnoprzepustowy} pierwszego rzędu (exponential moving average):

\begin{equation}
y_n = \alpha \cdot x_n + (1-\alpha) \cdot y_{n-1}
\end{equation}

gdzie $\alpha \in (0,1)$ to współczynnik wygładzania, $x_n$ to aktualny pomiar surowy, a $y_n$ to wartość filtrowana. Filtr ten tłumi szybkie zmiany sygnału, zachowując powolne zmiany pola magnetycznego.

\section{Planowana funkcjonalność}

\begin{enumerate}
    \item Ciągły pomiar trzech składowych pola magnetycznego (X, Y, Z)
    \item Obliczanie całkowitej wartości natężenia pola: $|\vec{B}| = \sqrt{B_x^2 + B_y^2 + B_z^2}$
    \item Kalibracja czujnika w celu eliminacji zakłóceń systematycznych
    \item Cyfrowa filtracja sygnału w celu redukcji szumów
    \item Wyświetlanie pomiarów w czasie rzeczywistym na ekranie OLED
    \item Wizualizacja danych przez interfejs szeregowy UART
\end{enumerate}

\section{Oczekiwane rezultaty}

Po zakończeniu projektu urządzenie powinno:
\begin{itemize}
    \item Wykrywać pole magnetyczne Ziemi z dokładnością $\pm 5$~$\mu$T
    \item Reagować na zbliżenie magnesów trwałych
    \item Stabilnie działać po kalibracji, eliminując lokalne zakłócenia
    \item Prezentować wyniki w czytelnej formie na wyświetlaczu
\end{itemize}

Projekt pozwoli na praktyczną weryfikację zjawisk elektromagnetycznych, w szczególności efektu Halla oraz właściwości pola magnetycznego Ziemi.

\end{document}
